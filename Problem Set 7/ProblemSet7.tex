\documentclass{article}
\usepackage{amsmath,amssymb,amsthm,latexsym,paralist}
\usepackage{array}
\usepackage{graphicx}
\usepackage{mathtools}
\usepackage{algpseudocode}
\usepackage{changepage}



\title{Problem Set 7}
\author{Parvesh Adi Lachman}
\date{November 2023}

\begin{document}
\maketitle
\section{Problem 1}

Consider the algorithm below, which takes an $n\geq 0$ and finds it remainder when divided by $c\geq 1$

\vspace{15pt}

\begin{algorithmic}
\Function{Remainder}{$n$}:
\If{$n\leq c-1$}
\State \Return $n$
\Else
\State \Return Remainder($n-c$)
\EndIf
\EndFunction
\end{algorithmic}

\vspace{10pt}


\noindent\textbf{Claim:} Let $c\geq 1$. For any $n\geq 0$, $\textbf{remainder($n$)}=n\mod c$.
\vspace{10pt}


\textbf{Step 0:} For $n\geq 0$, we want to show that $\textbf{remainder($n$)}=n\mod c$. 
\vspace{15pt}

\textbf{Step 1:} For any $n\geq 0$, Let $P(n)$ be the property that $\textbf{remainder($n$)}=n\mod c$.

\vspace{5pt}
We want to show that $P(n)$ is true for all $n\geq 0$.


\vspace{15pt}

\textbf{Step 2:} As base cases, consider when\vspace{10pt}


$n=0$. We will show that $P(0)$ is true: that is $\textbf{remainder($0$)}=0\mod c$. Fortunately, this is true since $c\geq 1$ and in the algorithm, if $n\leq c-1$, and in this case $n=0$, $c\geq1$. This implies $0\leq 1-1=0\leq0$ which is true, then $\textbf{remainder($n$)}=n$. Thus, $\textbf{remainder($0$)}=0$, so RHS = 0. Also, $0\mod c=0$, so LHS = 0. Thus, LHS = RHS, so $P(0)$ is true.
\vspace{10pt}

Additionally, for any $n\leq c-1$, $\textbf{remainder($n$)}=n=(n\mod c)$, this is true since due to the definition of the algorithm. 



\vspace{15pt}
\pagebreak
\textbf{Step 3:} Let $k\geq 1$. For the induction hypothesis, suppose that $P(0),\ldots,P(k)$ are true, or equivalently, that for all $0\leq k'\leq k\vcentcolon P(k')$. That is, suppose that $\textbf{remainder($k'$)}=k'\mod c$.
\vspace{15pt}

\textbf{Step 4:} Now we prove that $P(k+1)$ is true, using our induction assumptions that $P(0),\ldots,P(k)$ are true. That is, we prove that $\textbf{remainder($k+1$)}=(k+1)\mod c$.
\vspace{15pt}

\textbf{Step 5:} If $k + 1 < c$, then $\textbf{remainder($k + 1$)}=k+1=k+1\mod c$ by definition of algorithm. Otherwise the proof that $P(k+1)$ is true (given that $P(0),\ldots,P(k)$ are true) is as follows:
\vspace{5pt}



\begin{adjustwidth}{-1.5cm}{}
\noindent
\hspace{-3cm}
\begin{sloppypar}
	\begin{tabular}{l l l l}
		Left hand side of $P(k+1)$ & = & $\textbf{remainder($k+1$)}$ & \\
							  & = & $\textbf{remainder(($k+1)-c$)}$ & By def of algorithm, since $k+1\geq c-1$ \\
							  & = & $((k+1)-c)\mod c$ & By IH, since $0\leq (k+1)-c\leq k$ \\
							  & = & $(k+1)\mod c-c\mod c$ & By def of mod \\
							  & = & $(k+1)\mod c$ & Since $c\geq 1$, $c\mod c=0$ \\
							  & = & Right hand side of $P(k+1)$ & \\
	\end{tabular}
\end{sloppypar}
\end{adjustwidth}
\vspace{15pt}

\textbf{Step 6:} The steps above have shown that for any $k\geq 1$, if $P(0),\ldots,P(k)$ are true, then $P(k+1)$ is also true. Combined with the base cases, which show that $P(0)$ are true, we have shown that for all $n\geq 0$, $P(n)$ is true, as desired.

\pagebreak



\section{Problem 2}

\textbf{Claim:} Let $n,c\geq 1$ and $c\leq n$. The number of simple paths of length $c$ in the complete graph on $n$ nodes is $\frac{n!}{(n-c-1)!}$ which is equal to $n(n-1)\cdots (n-c)$.

\vspace{10pt}

\textbf{complete graph} $K_n$: an undirected graph on $n$ nodes with an edge between every pair of nodes.
\vspace{5pt}

\textbf{simple path:} a sequence of distinct nodes with edges between consecutive nodes in the sequence.
\vspace{5pt}

\textbf{length of a path:} the number of \textit{edges} in the path (\textbf{not} number of nodes).
\vspace{5pt}


\textbf{Step 0:} For all $c\geq 1$, we want to show that the number of simple paths of length $c$ in the complete graph on $n$ nodes is $\frac{n!}{(n-c-1)!}$ which is equal to $n(n-1)\cdots (n-c)$.
\vspace{15pt}

\textbf{Step 1:} For any $c\geq 1$, Let $P(c)$ be the property that the number of simple paths of length $c$ in the complete graph on $n$ nodes is $\frac{n!}{(n-c-1)!}$ which is equal to $n(n-1)\cdots (n-c)$.
\vspace{15pt}

\textbf{Step 2:} As base cases, consider when $c=1$. We will show that $P(1)$ is true: that is the number of simple paths of length $1$ in the complete graph on $n$ nodes is $\frac{n!}{(n-1-1)!}$ which is equal to $n(n-1)$. Fortunately, this is true since the number of simple paths of length $1$ in the complete graph on $n$ nodes is $n(n-1)$, and $\frac{n!}{(n-1-1)!}=n!$, and $n!=n(n-1)$. Thus, the number of simple paths of length $1$ in the complete graph on $n$ nodes is $\frac{n!}{(n-1-1)!}$ which is equal to $n(n-1)$, so LHS = RHS. Thus, $P(1)$ is true.
\vspace{15pt}


\textbf{Step 3:} For the induction hypothesis, suppose (hypothetically) that $P(k)$ is true for some fixed $k\geq 1$. That is, suppose that the number of simple paths of length $k$ in the complete graph on $n$ nodes is $\frac{n!}{(n-k-1)!}$ which is equal to $n(n-1)\cdots (n-k)$.\vspace{15pt}

\textbf{Step 4:} Now we prove that $P(k+1)$ is true, using our induction assumptions that $P(k)$ is true. That is, we prove that the number of simple paths of length $k+1$ in the complete graph on $n$ nodes is $\frac{n!}{(n-(k+1)-1)!}$ which is equal to $n(n-1)\cdots (n-(k+1))$.\vspace{15pt}


\textbf{Step 5:} The proof that $P(k+1)$ is true (given that $P(k)$ is true) is as follows:\vspace{15pt}


\begin{adjustwidth}{-1.5cm}{}
	\begin{sloppypar}
		\begin{tabular}{l l l l}
			Left hand side of $P(k+1)$ & = & The number of simple paths of length $k+1$ in $K_n$ & \\
									   & = & The number of simple paths of length $k$ in $K_n$ * $(n-(k+1))$ & Since there \\
									   &  & & are $k+1$ nodes \\
									   & & & in length $k$ path \\
									   & = & $\frac{n!}{(n-k-1)!}\cdot (n-(k+1))$ & By IH \\
									   & = & $\frac{n!}{(n-k-1)!}\cdot \frac{(n-k-1)!}{(n-(k+1)-1)!}$ & By algebra \\
									   & = & $\frac{n!}{(n-(k+1)-1)!}$ & By algebra \\
									   & = & Right hand side of $P(k+1)$ & \\
		\end{tabular}
	\end{sloppypar}
\end{adjustwidth}
\vspace{15pt}

\textbf{Step 6:} The steps above have shown that if $P(k)$ is true, then $P(k+1)$ is also true. Combined with the base case, which shows that $P(1)$ is true, we have shown that for all $c\geq 1$, $P(c)$ is true, as desired.



\pagebreak

\section{Problem 3}

Recall the Fibonacci numbers, as defined by:
\begin{center}
    \begin{minipage}{0.5\textwidth} % Adjust the width as needed
        \raggedright % Align text to the left
        $f_1=1$ \\
        $f_2=1$ \\
        $f_n=f_{n-1}+f_{n-2}$ for $n\geq 3$
    \end{minipage}
\end{center}
\vspace{10pt}

\noindent Recall the Sharp numbers from PS6, as defined by:
\begin{center}
    \begin{minipage}{0.5\textwidth} % Adjust the width as needed
        \raggedright % Align text to the left
		$s_1=2$ \\
		$s_2=4$ \\
		$s_n=s_{n-1}+s_{n-2}$ for $n\geq 3$
    \end{minipage}
\end{center}
\vspace{15pt}

\noindent\textbf{Claim:} For all $n\geq 3$, $s_n=4\cdot f_{n-1}+2\cdot f_{n-2}$.

\vspace{15pt}

\noindent\textbf{Step 0:} For $n\geq 3$, we want to show that $s_n=4\cdot f_{n-1}+2\cdot f_{n-2}$.
\vspace{15pt}

\noindent\textbf{Step 1:} For any $n\geq 3$, Let $P(n)$ be the property that $s_n=4\cdot f_{n-1}+2\cdot f_{n-2}$.
\vspace{5pt}
We want to show that $P(n)$ is true for all $n\geq 3$.

\vspace{15pt}

\noindent\textbf{Step 2:} As base cases, consider when\vspace{10pt}

$n=3$. We will show that $P(3)$ is true: that is $s_3=4\cdot f_{3-1}+2\cdot f_{3-2}$. Fortunately, this is true since $s_3=s_2+s_1=4+2=6$ and $4\cdot f_{3-1}+2\cdot f_{3-2}=4\cdot f_2+2\cdot f_1=4\cdot 1+2\cdot 1=6$. Thus, $s_3=4\cdot f_{3-1}+2\cdot f_{3-2}$, so LHS = RHS. Thus, $P(3)$ is true.

\vspace{5pt}
$n=4$. We will show that $P(4)$ is true: that is $s_4=4\cdot f_{4-1}+2\cdot f_{4-2}$. Fortunately, this is true since $s_4=s_3+s_2=6+4=10$ and $4\cdot f_{4-1}+2\cdot f_{4-2}=4\cdot f_3+2\cdot f_2=4\cdot 2+2\cdot 1=10$. Thus, $s_4=4\cdot f_{4-1}+2\cdot f_{4-2}$, so LHS = RHS. Thus, $P(4)$ is true.


\vspace{15pt}


\noindent\textbf{Step 3:} Let $k\geq 4$. For the induction hypothesis, suppose that $P(3),P(4),\ldots,P(k)$ are true, or equivalently, that for all $3\leq k'\leq k\vcentcolon P(k')$. That is, suppose that $s_{k'}=4\cdot f_{k'-1}+2\cdot f_{k'-2}$.
\vspace{15pt}

\noindent\textbf{Step 4:} Now we prove that $P(k+1)$ is true, using our induction assumptions that $P(3),P(4),\ldots,P(k)$ are true. That is, we prove that $s_{k+1}=4\cdot f_{k+1-1}+2\cdot f_{k+1-2}$.

\vspace{15pt}


\noindent\textbf{Step 5:} The proof that $P(k+1)$ is true (given that $P(3),P(4)\ldots,P(k)$ are true) is as follows:
\vspace{5pt}

\begin{adjustwidth}{-1.5cm}{}
\noindent
\hspace{-3cm}
\begin{sloppypar}
	\begin{tabular}{l l l l}
		Left hand side of $P(k+1)$ & = & $s_{k+1}$ & \\
								   & = & $s_{k}+s_{k-1}$ & By def of sequence \\
								   &  & & (Sharp numbers)\\
								   & = & $(4\cdot f_{k-1}+2\cdot f_{k-2})+(4\cdot f_{k-2}+2\cdot f_{k-3})$ & By IH \\
								   & = & $4\cdot f_{k-1}+2\cdot f_{k-2}+4\cdot f_{k-2}+2\cdot f_{k-3}$ & By algebra\\
								   & = & $4\cdot f_{k-1}+4\cdot f_{k-2}+2\cdot f_{k-2}+2\cdot f_{k-3}$ & By rewriting \\
								   & = & $4(f_{k-1}+f_{k-2}) + 2(f_{k-2}+f_{k-3})$ & By factoring and algebra \\
								   & = & $4\cdot f_{k} + 2\cdot f_{k-1}$ & By def of sequence \\
								   &   & & (Fibonacci numbers)\\
								   & = & Right hand side of $P(k+1)$ & \\
								   
	\end{tabular}
\end{sloppypar}
\end{adjustwidth}
\vspace{15pt}

\textbf{Step 6:} The steps above have shown that for any $k\geq 4$, if $P(3),P(4),\ldots,P(k)$ are true, then $P(k+1)$ is also true. Combined with the base cases, which show that $P(3),P(4)$ are true, we have shown that for all $n\geq 3$, $P(n)$ is true, as desired.




\end{document}

