\documentclass{article}
\usepackage{amsmath,amssymb,amsthm,latexsym,paralist}
\usepackage{graphicx}
\usepackage{array}
\usepackage{mathtools}
\usepackage{algpseudocode}
\usepackage{changepage}

%\title{Test 2}
%\author{Parvesh Adi Lachman}
%\date{November 2023}

\begin{document}

\begingroup  
  \centering
  \LARGE Test 2\\[1.25em]
  \large Parvesh Adi Lachman\par\vspace{5pt}
  \large November 2023\par
\endgroup


\section{Problem 1}
\noindent\textbf{Claim:} Let $G=(W,E)$ be a (simple, undirected) graph with $|W|\geq 5$. Then there are two distinct subsets of nodes $U,V\subseteq W$ with $|U| = |V| =2$ such that $\sum_{x\in U} deg(x)=\sum_{x\in V} deg(x)$.\vspace{15pt}

\noindent\textbf{Proof:} We will prove this claim using the Pigeonhole Principle. 

\begin{itemize}
	\item Let the pigeons ($A$) be the set of all subsets of $W$ with exactly two elements.
	\item Let the pigeonholes ($B$) be the set of all possible sums of degrees of nodes in subsets of $W$ with exactly two elements. Thus, $B\vcentcolon =\{0,1,2,...,2|W|-1\}$
	\item Let $f\vcentcolon A\rightarrow B$ be defined by $f(T)\vcentcolon =\sum_{x\in T} deg(x)$. Note that $f$ is a well defined function:
		\begin{enumerate}[(1)]
			\item for any pigeon $T\in A$, $f(T)=\sum_{x\in T} deg(x)$ is computable because $T$ is a finite set and $deg(x)$ is defined for all $x\in W$.
			\item for any pigeon $T\in A$, if $f(T_1)=b$ and $f(T_2)=c$ then $b=c$ because $f(T_1)=\sum_{x\in T_1} deg(x)$ and $f(T_2)=\sum_{x\in T_2} deg(x)$, and since $T_1$ and $T_2$ are subsets of $W$, $\sum_{x\in T_1} deg(x)=\sum_{x\in T_2} deg(x)$. 
			\item for any pigeon $T\in A$, $f(T)=\sum_{x\in T} deg(x)$ is within the codomain $B$ because $T$ is a subset of $W$ and $deg(x)$ is defined for all $x\in W$.
		\end{enumerate}	
\end{itemize}

\begin{flushleft}
	\begin{tabular}{|p{0.5cm}|p{6.6cm}|p{5.5cm}|}
		\hline
		 & \textbf{Mathematical Reasoning} & \textbf{Reason this Statement is True (From the Approved List)} \\
		\hline
		$\implies$ & $|A|= \binom{|W|(|W|-1)}{2}$, where $|W|\geq 5$ & Because $A=\binom{|W|}{2}$ \\
		\hline
		$\implies$ & $|B|= 2|W|-1$, where $|W|\geq 5$& Because $B=\{0,1,2,...,2|W|-1\}$ \\
		\hline
		$\implies$ & $|A|>|B|$ & Since $\binom{|W|(|W|-1)}{2}> 2|W|-1$ \\
		\hline
		$\implies$ & $\exists a_1a_2\in A\vcentcolon [(a_1\neq a_2)\land (f(a_1)=f(a_2))]$ & By the Pigeonhole Principle \\
		\hline
		$\implies$ & $\exists U,V\subseteq W\vcentcolon [(U\neq V)\land (|U|=|V|=2)\land (\sum_{x\in U} deg(x)=\sum_{x\in V} deg(x))]$ & By def of $A$ and $f$ \\
		\hline
	\end{tabular}
\end{flushleft}\vspace{15pt}

\qed



\pagebreak

\section{Problem 2}

\noindent Recall the following definitons from lecture about a function $g\vcentcolon A\rightarrow B\vcentcolon$\vspace{10pt}

\vspace{10pt}
\noindent\textbf{one to one:} $\forall n,m\in A\vcentcolon (n\neq m)\implies (g(n)\neq g(m))$

\vspace{10pt}
\noindent\textbf{onto:} $\forall b\in B\vcentcolon \exists a\in A\vcentcolon g(a)=b$
\vspace{10pt}


\noindent\textbf{Claim:} Let $f\vcentcolon\mathbb{N}\rightarrow\mathbb{Z}$ be defined by $f(n)\vcentcolon = \sum_{v\in K_n} deg(v)$, where $K_n$ is the complete graph on $n$ nodes.
\vspace{10pt}

(a) For all $x\in S$, we want to show that if $P(x)$ is true then $Q(x)$ is also true.\vspace{10pt}

(b) \hspace{5pt} $\neg Q(x)\implies\neg P(x)$. \vspace{10pt}


(c) To prove $P(x)\implies Q(x)$ by contrapositive we assume $\neg Q(x)$ and use that to show $\neg P(x)$. \vspace{10pt}

(d) $\forall n,m\in\mathbb{N}\vcentcolon (n\neq m)\implies (f(n)\neq f(m))$ \vspace{10pt}

(e) \textbf{Proof:} We want to show $\forall n,m\in\mathbb{N}\vcentcolon (n\neq x)\implies (f(n)\neq f(m))$. To prove this statement, we want to show that for any $n,m\in\mathbb{N}$, if $n\neq m$ then $f(n)\neq f(m)$.  We will prove the equivalent contrapositive, that is, $\exists n,m\in\mathbb{N}\vcentcolon f(n)=f(m)\implies n=m$.  To do this we will assume $f(n)=f(m)$ and use that to show $n=m$. \vspace{10pt} 

\begin{flushleft}
	\begin{tabular}{|p{0.5cm}|p{5.8cm}|p{6.2cm}|}
		\hline
		 & \textbf{Mathematical Reasoning} & \textbf{Reason this Statement is True (From the Approved List)} \\
		\hline
		$\implies$ & $f(n)=f(m)$ & Given \\
		\hline
		$\implies$ & $\sum_{v\in K_n} \text{deg}(v) = \sum_{v\in K_m} \text{deg}(v)$ & By the definition of $f$ \\
		\hline
		$\implies$ & $n(n-1)=m(m-1)$ & By the definition of $K_n$ and $K_m$, WLOG $\sum_{v\in K_n} deg(v)=2|E|$, where $|E| = \frac{n(n-1)}{2}$, thus by algebra $\sum_{v\in K_n} deg(v)=|E|$\vspace{5pt} \\
		\hline
		$\implies$ & $n^2-n=m^2-m$ & By algebra \\
		\hline
		$\implies$ & $n^2-n-m^2+m=0$ & By algebra \\
		\hline
		$\implies$ & $(n-m)(n+m-1)=0$ & Since $n^2-n-m^2+m=(n-m)(n+m-1)$ (By algebra) \\
		\hline
		$\implies$ & $n-m=0$ & Since $n,m\in\mathbb{N}$, $n+m-1\neq 0$ \\
		\hline
		$\implies$ & $n=m$ & By algebra \\
		\hline

	\end{tabular}
\end{flushleft}\vspace{15pt}

Thus, we have shown that for any $n,m\in\mathbb{N}$, if $f(n)=f(m)$ then $n=m$. \qed\vspace{15pt}

(f)
\textbf{Counter Example:} To prove $f$ is not onto, we want to show the negation of the definition of onto. That is, we want to show that $\exists b\in\mathbb{Z}\vcentcolon \forall a\in\mathbb{N}\vcentcolon f(a)\neq b$. Consider $b=0,b\in\mathbb{Z}$.  In a complete graph $K_n$, there exists no $f(a)=\sum_{v\in K_a} deg(v)$.  Meaning there is no $a\in\mathbb{N}$ that $a$ maps to $b\in\mathbb{Z}$.  Thus $f$ is not onto. \vspace{10pt} 





\pagebreak

\section{Problem 3}
\noindent Consider the following sequence of numbers similar to (But not the same as) the Sharp numbers.
\begin{center}
	\begin{minipage}{0.5\textwidth}
		$d_1=2$ \\
		$d_2=4$ \\
		$d_n=d_{n-1}+2\cdot d_{n-2}$, for $n\geq 3$ \\
	\end{minipage}
\end{center}

\vspace{10pt}

\noindent\textbf{Claim:} For all $n\geq 1$, $d_n=2^n$\vspace{20pt}


\noindent\textbf{Step 0:} For all $n\geq 1$, we want to show that $d_n=2^n$.\vspace{15pt}

\noindent\textbf{Step 1:} For any $n\geq 1$, let $P(n)$ be the property that $d_n=2^n$. We want to show $\forall n\geq 1\vcentcolon P(n)$. \vspace{15pt}

\noindent\textbf{Step 2:} As base cases consider when\vspace{7pt}

$n=1$. We will show that $P(1)$ is true: that is, that $d_1=2^1$. Fortunately,

\begin{center}
	left hand side $=d_1=2=2^1=$ right hand side \\ 
\end{center}

$n=2$. We will show that $P(2)$ is true: that is, that $d_2=2^2$. Fortunately,

\begin{center}
	left hand side $=d_2=4=2^2=$ right hand side \\
\end{center}


\noindent\textbf{Step 3:} Let $k\geq 2$. For the induction hypothesis, suppose that $P(1),..,P(k)$ are true, or equivalently, that for all $1\leq k' \leq k\vcentcolon P(k')$. That is, suppose that 

\begin{center}
	$\forall 1\leq k'\leq k\vcentcolon d_{k'}=2^{k'}$
\end{center}
\vspace{15pt}


\noindent\textbf{Step 4:} Now we prove that $P(k+1)$ is true, using our induction assumptions that $P(1),..,P(k+1)$ are true. That is, we prove that
\begin{center}
	$d_{k+1}=2^{k+1}$ \\
\end{center}
\vspace{15pt}


\noindent\textbf{Step 5:} The proof that $P(k+1)$ is true (given that $P(1),...,P(k)$ are true) is as follows:
\vspace{5pt}

\begin{adjustwidth}{-0.25cm}{}
	\noindent
	\hspace{-3cm}
	\begin{sloppypar}
		\begin{tabular}{l l l l}
			Left hand side of $P(k)$ & $=$ & $d_{k+1}$ & \\
									 & $=$ & $d_{k}+2\cdot d_{k-1}$ & By def of sequence \\
									 & $=$ & $2^k+2\cdot 2^{k-1}$ & By IH since $1\leq k-1\leq k$\\
									 & $=$ & $2^k+2^k$ & By algebra \\
									 & $=$ & $2\cdot 2^k$ & By algebra \\
									 & $=$ & $2^{k+1}$ & By algebra \\
									 & $=$ & Right hand side of $P(k+1)$ & \\
		\end{tabular}
	\end{sloppypar}
	
\end{adjustwidth}
\vspace{15pt}

\noindent\textbf{Step 6:} The steps above have shown that for any $k\geq 2$, if $P(1),...,P(k)$ are true, then $P(k+1)$ is also true. Combined with the base cases which show that $P(1)$ and $P(2)$ are true, we have shown that for all $n\geq 1$, $P(n)$ is true, as desired. \qed




\end{document}
