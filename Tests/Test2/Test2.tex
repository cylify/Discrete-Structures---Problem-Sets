\documentclass{article}
\usepackage{amsmath,amssymb,amsthm,latexsym,paralist}
\usepackage{graphicx}
\usepackage{array}
\usepackage{mathtools}
\usepackage{algpseudocode}
\usepackage{changepage}




\title{Test 2}
\author{Parvesh Adi Lachman}
\date{November 2023}

\begin{document}
\maketitle

\section{Problem 1}








\section{Problem 2}

\noindent Recall the following definitons from lecture about a function $g\vcentcolon A\rightarrow B\vcentcolon$\vspace{10pt}

\vspace{10pt}
\noindent\textbf{one to one:} $\forall n,m\in A\vcentcolon (n\neq m)\implies (g(n)\neq g(m))$

\vspace{10pt}
\noindent\textbf{onto:} $\forall b\in B\vcentcolon \exists a\in A\vcentcolon g(a)=b$
\vspace{10pt}


\noindent Let $f\vcentcolon\mathbb{N}\rightarrow\mathbb{Z}$ be defined by $f(n)\vcentcolon = \sum_{v\in K_n} deg(v)$, where $K_n$ is the complete graph on $n$ nodes.
\vspace{10pt}

(a) Suppose you are trying to prove a statement of the form $\forall x\in S\vcentcolon [P(x)\implies Q(x)]$.  What is the first line of this "for all" proof, as we've seen in this courses? \vspace{10pt} 


(b) Suppose you are trying to prove a statement of the form $P(x)\implies Q(x)$. What is the contrapositive of this claim? \vspace{10pt}


\hspace{5pt} $\neg Q(x)\implies\neg P(x)$. \vspace{10pt}




\pagebreak

\section{Problem 3}
\noindent Consider the following sequence of numbers similar to (But not the same as) the Sharp numbers.
\begin{center}
	\begin{minipage}{0.5\textwidth}
		$d_1=2$ \\
		$d_2=4$ \\
		$d_n=d_{n-1}+2\cdot d_{n-2}$, for $n\geq 3$ \\
	\end{minipage}
\end{center}

\vspace{10pt}

\noindent\textbf{Claim:} For all $n\geq 1$, $d_n=2^n$\vspace{20pt}


\noindent\textbf{Step 0:} For all $n\geq 1$, we want to show that $d_n=2^n$.\vspace{15pt}

\noindent\textbf{Step 1:} For any $n\geq 1$, let $P(n)$ be the property that $d_n=2^n$. We want to show $\forall n\geq 1\vcentcolon P(n)$. \vspace{15pt}

\noindent\textbf{Step 2:} As base cases consider when\vspace{7pt}

$n=1$. We will show that $P(1)$ is true: that is, that $d_1=2^1$. Fortunately,

\begin{center}
	left hand side $=d_1=2=2^1=$ right hand side \\ 
\end{center}

$n=2$. We will show that $P(2)$ is true: that is, that $d_2=2^2$. Fortunately,

\begin{center}
	left hand side $=d_2=4=2^2=$ right hand side \\
\end{center}


\noindent\textbf{Step 3:} Let $k\geq 2$. For the induction hypothesis, suppose that $P(1),..,P(k)$ are true, or equivalently, that for all $1\leq k' \leq k\vcentcolon P(k')$. That is, suppose that 

\begin{center}
	$\forall 1\leq k'\leq k\vcentcolon d_{k'}=2^{k'}$
\end{center}
\vspace{15pt}


\noindent\textbf{Step 4:} Now we prove that $P(k+1)$ is true, using our induction assumptions that $P(1),..,P(k+1)$ are true. That is, we prove that
\begin{center}
	$d_{k+1}=2^{k+1}$ \\
\end{center}
\vspace{15pt}


\noindent\textbf{Step 5:} The proof that $P(k+1)$ is true (given that $P(1),...,P(k)$ are true) is as follows:
\vspace{5pt}

\begin{adjustwidth}{-0.25cm}{}
	\noindent
	\hspace{-3cm}
	\begin{sloppypar}
		\begin{tabular}{l l l l}
			Left hand side of $P(k)$ & $=$ & $d_{k+1}$ & \\
									 & $=$ & $d_{k}+2\cdot d_{k-1}$ & By def of sequence \\
									 & $=$ & $2^k+2\cdot 2^{k-1}$ & By IH \\
									 & $=$ & $2^k+2^k$ & By algebra \\
									 & $=$ & $2\cdot 2^k$ & By algebra \\
									 & $=$ & $2^{k+1}$ & By algebra \\
									 & $=$ & Right hand side of $P(k+1)$ & \\
		\end{tabular}
	\end{sloppypar}
	
\end{adjustwidth}
\vspace{15pt}

\noindent\textbf{Step 6:} The steps above have shown that for any $k\geq 2$, if $P(1),...,P(k)$ are true, then $P(k+1)$ is also true. Combined with the base cases which show that $P(1)$ and $P(2)$ are true, we have shown that for all $n\geq 1$, $P(n)$ is true, as desired. \qed




\end{document}
